\begin{abstract}
\textit{
	A high level of security, to the extent that all hacks are rendered impossible,
	is of crucial importance in most software. Obfuscation is a potential method
	of protecting sensitive information, whereby the computer program is rendered
	unintelligible to a human being, without altering the core functionality of
	the program. As in any secure system, obfucators increase the work an attacker
	needs to perform, however a potential tradeoff in the use of obfuscation is that
	it may impact the execution speed of a program, especially if the
	program size exhausts the CPU cache capacity, and additional calls to main memory
	are required.
	To test the relationship between obfuscation and execution speed, we modelled
	a benchmark program using a fixed-width boolean circuit, where the
	width of the circuit is increased with each execution. We found that execution
	speed decreases as circuit width increases, with a marked increase in the
	execution time when the circuit width becomes wider than the CPU's highest cache
	level. However, the latency increase due to the shift to main memory is markedly
	less than was anticipated, the causes of which need to be further investigated.
}
\end{abstract}
