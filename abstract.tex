\begin{abstract}
\textit{
	The optimisation of software code execution speed is a key consideration in any software project.
	A high level of security, to the extent that all ‘hacks’ are rendered impossible, is also of crucial importance to most software. One way of protecting sensitive information is through obfuscation, which aims to make the computer programme unintelligible to a human being, without altering the core functionality of the programme. 
	A problem with obfuscation, is that it may slow down the processing speed of a software, and therefore reduce the overall user experience.
	We analyse the results of a preliminary experiment that benchmarks the execution time of a fixed-width boolean circuit, where the width of the circuit is increased with each execution. The circuit represents an approximation of a result of a functional encryption operator, i.e. of an obfuscation. The experiment collects cache data generated by the executing system’s hardware performance counters, to measure the execution time of the circuit.
}
\end{abstract}
