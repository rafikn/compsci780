\section{Conclusions and Future Work}
Obfuscation is a potentially powerful tool to encrypt and protect data in the future. However, one of the limitations of obfuscation may prove to be that it lowers the execution speed of a programme, at least in some circumstances.
\par
In this experiment we simulated an obfuscated circuit by generating a fixed-width random boolean circuit and evaluated the performance of that circuit. We have shown how \textit{perf} hardware counters can be used to measure cache hits inside a 6MB L3-cache ix86 microprocessor clocked at 4GHz in order to evaluate a simulated obfuscated circuit.
\par
We have proposed an analytical model to explain the execution times observed. This analytical model holds true even when the circuit width is very large, i.e. where $w > 10^6$. Further work is required to determine whether this analytical model can predict execution speed to a statistically significant degree of accuracy. In this study our assumption as that when $w$ is large, it would be possible to seperate tg and tm, as tg would become negligible as tm would increase significantly and dominate the results. However the results obtained failed to match this prediction, as the fixed part of T (...) was of a similar order of magnitude as the variable component.
\par
Future work would explore the use of the Intel Performance Counter Monitor library\footnote{\url{https://software.intel.com/en-us/articles/intel-performance-counter-monitor}} as a high level interface for measuring real time performance data directly inside the circuit's code to seperate . In particular, we plan to measure the CPU cache and instruction metrics at initialisation time, then at generation time and finally at execution time to isolate each stage's impact on performace.

A more extensive study, using a large range of hardware devices and over a large range of circuit simulations, could help validate or disprove the proposed model.
\par
We are focusing on improving our simulation by replacing the random generation mechanism using the \textit{C} standard library \textit{rand()} method with the ’stride’ approach\cite{stride} to 
\par
 

