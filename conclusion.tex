\section{Conclusions and Future Work}
Obfuscation is a potentially powerful tool to encrypt and protect data in the future. However, one of the limitations of obfuscation may prove to be that it lowers the execution speed of a programme, at least in some circumstances.
\par
In this experiment we simulated an obfuscated circuit, and evaluated the performance of that circuit. We have shown how perf hardware counters can be used to measure cache hits inside a 6MB L3-cache ix86 microprocessor clocked at 3GHz in order to evaluate a simulated obfuscated circuit.
\par
Based on the execution times observed in the experiment, we have proposed an analytical model to explain the execution times. This analytical model holds true even when the circuit width is very large, i.e. where $w > 10^6$ . Further work is required to determine whether this analytical model can predict execution speed to a statistically significant degree of accuracy. A more extensive study, using a large range of hardware devices and over a large range of circuit simulations, could help validate or disprove the proposed model.
\par
We plan to improve our simulation by replacing the random generation mechanism using the C random function with the ’stride’ method described in this paper. We also plan to use the Intel Performance Counter Monitor library as a high level interface for measuring real time performance data directly inside the code. In particular, we plan to measure the CPU cache and instruction metrics at initialisation time, then at generation time and finally at execution time. This will help make life better for everyone.